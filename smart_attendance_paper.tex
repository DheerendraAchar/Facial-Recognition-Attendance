
\documentclass[conference]{IEEEtran}
\IEEEoverridecommandlockouts
\usepackage{graphicx}
\usepackage{amsmath}

\begin{document}

\title{Smart Attendance System using Facial Recognition and Attention Tracking}

\author{
\IEEEauthorblockN{B Dheerendra Achar, Chiranjeev Kapoor, Gaurav Bhandare, Harsh Dwivedi}
\IEEEauthorblockA{Department of Computer Science and Engineering \\
Dayananda Sagar University, Bengaluru, India \\
\{dheerendra.achar, chiranjeev.kapoor, gaurav.bhandare, harsh.dwivedi\}@dsc.edu.in}
\and
\IEEEauthorblockN{Prof. Dharmendra D.P}
\IEEEauthorblockA{Assistant Professor \\
Department of Computer Science and Engineering \\
Dayananda Sagar University, Bengaluru, India \\
dharmendra.dp@dsc.edu.in}
}

\maketitle

\begin{abstract}
This paper presents a novel Smart Attendance System that integrates facial recognition with real-time attention tracking to ensure robust and proxy-resistant attendance management in educational settings. Traditional attendance systems are vulnerable to proxy marking and suffer from low accuracy in uncontrolled environments. Our proposed system leverages deep learning-based face detection (MTCNN), dlib-based facial alignment, and face encoding (face\_recognition) for accurate identification. Furthermore, we implement attention tracking using head pose estimation to verify student attentiveness over time. A multi-frame attention buffer and a visual progress bar ensure that only students who maintain continuous attention are marked present. A Streamlit dashboard offers real-time insights into attendance logs and attention trends. The system demonstrates improved accuracy, proxy resistance, and ease of integration in classroom settings.
\end{abstract}

\begin{IEEEkeywords}
Facial Recognition, Attendance System, Attention Tracking, Head Pose Estimation, Computer Vision, Human-Computer Interaction, MTCNN, dlib, Streamlit
\end{IEEEkeywords}

\section{Introduction}
Traditional attendance methods in educational institutions are time-consuming and prone to errors or proxy marking. With the advent of computer vision and deep learning, facial recognition-based attendance systems have gained popularity. However, many existing solutions fail to account for student engagement, often marking students present even when they are inattentive or absent. Our system addresses this limitation by introducing attention-aware attendance verification.

\section{Related Work}
Facial recognition for attendance has been explored using CNNs, Haar cascades, and deep metric learning techniques such as FaceNet and ArcFace. While these methods achieve reasonable accuracy, they often neglect attention validation. Previous studies have proposed gaze tracking and head pose estimation for engagement monitoring, but integration into attendance workflows remains limited. Our system bridges this gap by combining real-time face recognition with continuous attention tracking for robust attendance marking.

\section{Proposed System}
The system architecture comprises four key modules:
\begin{enumerate}
    \item \textbf{Face Detection and Alignment:} We use MTCNN for accurate multi-face detection, followed by dlib's 68-point landmark model for facial alignment. This ensures consistent embeddings under varying poses.
    \item \textbf{Face Encoding and Recognition:} The aligned faces are encoded using the \textit{face\_recognition} library based on deep ResNet models. Average embeddings per student are stored for matching.
    \item \textbf{Attention Tracking:} Head pose estimation is performed using dlib landmarks and OpenCV's solvePnP to calculate pitch, yaw, and roll angles. A multi-frame buffer (30 frames) ensures students must maintain attention (pitch, yaw within $\pm$30$^\circ$) for approximately 2 seconds to be marked present. Color-coded visual progress bars provide real-time feedback (green/yellow/red based on attention score).
    \item \textbf{Dashboard and Analytics:} A Streamlit-based dashboard visualizes attendance logs, attention trends, and allows CSV export. Date filters and per-student analytics enable educators to track engagement over time.
\end{enumerate}

\section{Implementation}
The system was developed in Python 3.10 on macOS, using libraries such as OpenCV, dlib, face\_recognition, MTCNN, and Streamlit. Facial images are captured at runtime and compared against a dataset of known students. Attention scores are computed per frame and logged in a CSV file alongside timestamps. The dashboard provides interactive visualizations and filtering capabilities.

\section{Results and Evaluation}
Initial testing in a controlled classroom setup demonstrated:
\begin{itemize}
    \item 94.2\% recognition accuracy for enrolled students.
    \item Near-zero false positives due to attention filtering.
    \item Proxy prevention by rejecting inattentive or transient faces.
\end{itemize}
Future evaluations will involve larger datasets and multi-camera setups.

\section{Conclusion and Future Work}
Our system demonstrates an effective solution for attendance automation by combining facial recognition with attention monitoring. Future extensions include support for multi-camera classrooms, integration with LMS platforms, and deployment on edge devices. We also aim to explore gaze tracking for more granular engagement analysis.

\section*{Acknowledgment}
We thank Prof. Dharmendra D.P for his invaluable guidance throughout this project.

\bibliographystyle{IEEEtran}
\begin{thebibliography}{1}

\bibitem{facenet}
F. Schroff, D. Kalenichenko, and J. Philbin, ``FaceNet: A unified embedding for face recognition and clustering,'' in \textit{Proc. IEEE Conf. Comput. Vis. Pattern Recognit. (CVPR)}, 2015, pp. 815--823.

\bibitem{dlib}
D. King, ``Dlib-ml: A machine learning toolkit,'' \textit{J. Mach. Learn. Res.}, vol. 10, pp. 1755--1758, 2009.

\bibitem{mtcnn}
K. Zhang, Z. Zhang, Z. Li, and Y. Qiao, ``Joint face detection and alignment using multi-task cascaded convolutional networks,'' \textit{IEEE Signal Process. Lett.}, vol. 23, no. 10, pp. 1499--1503, 2016.

\end{thebibliography}

\end{document}
